\documentclass[11pt,a4paper]{article}
\usepackage[utf8]{inputenc}
\usepackage[english,russian]{babel}
\usepackage{indentfirst}
\usepackage{misccorr}
\usepackage{graphicx}
\usepackage{amsmath}
\usepackage{verbatim}
\usepackage{sverb}
\begin{document}
	\begin{center}
	\thispagestyle{empty}
	{\LARGE Университет ИТМО, факультет ПИиКТ}
	\end{center}
	
	\vspace{6em}
	\begin{center}
		{\LARGE Лабораторная работа №2 по \\“Информационные системы и базы данных”\\Вариант:2587}
	\end{center}
	
	\vspace{20em}
	\begin{flushright}
		Выполнил: Яремко Р.О.\\
		Группа: P33113\\
		Преподаватель: Харитонова А.Е.
	\end{flushright}
	
	\vspace{\fill}
	\begin{center}
		Санкт-Петербург\\2020г.
	\end{center}
	
	\newpage
	\begin{flushleft}
		\textbf{{\large Текст задания:}}
		Составить запросы на языке SQL(пункты 1-7).
		\begin{enumerate}
			\item Сделать запрос для получения атрибутов из указанных таблиц, применив фильтры по указанным условиям:\\
				Таблицы: Н\_ЛЮДИ, Н\_СЕССИЯ.\\
				Вывести атрибуты: Н\_ЛЮДИ.ФАМИЛИЯ, Н\_СЕССИЯ.УЧГОД.\\
				Фильтры (AND):\\
				a) Н\_ЛЮДИ.ИМЯ = Александр.\\
				b) Н\_СЕССИЯ.УЧГОД = 2001/2002.\\
				Вид соединения: LEFT JOIN.
			\item Сделать запрос для получения атрибутов из указанных таблиц, применив фильтры по указанным условиям:\\
				Таблицы: Н\_ЛЮДИ, Н\_ОБУЧЕНИЯ, Н\_УЧЕНИКИ.\\
				Вывести атрибуты: Н\_ЛЮДИ.ИД, Н\_ОБУЧЕНИЯ.ЧЛВК\_ИД, Н\_УЧЕНИКИ.ИД.\\
				Фильтры: (AND)\\
				a) Н\_ЛЮДИ.ОТЧЕСТВО = Александрович.\\
				b) Н\_ОБУЧЕНИЯ.ЧЛВК\_ИД > 112514.\\
				c) Н\_УЧЕНИКИ.ИД < 100410.\\
				Вид соединения: RIGHT JOIN.
			\item Составить запрос, который ответит на вопрос, есть ли среди студентов группы 3102 те, кто старше 25 лет.
			\item В таблице Н\_ГРУППЫ\_ПЛАНОВ найти номера планов, по которым обучается (обучалось) более 2 групп на кафедре вычислительной техники.\\
				Для реализации использовать подзапрос.
			\item Выведите таблицу со средним возрастом студентов во всех группах (Группа, Средний возраст), где средний возраст меньше минимального возраста в группе 1101.
			\item Получить список студентов, зачисленных до первого сентября 2012 года на первый курс очной или заочной формы обучения (специальность: 230101). В результат включить:\\
				номер группы;\\
				номер, фамилию, имя и отчество студента;\\
				номер и состояние пункта приказа;\\
				Для реализации использовать соединение таблиц.
			\item Вывести список людей, не являющихся или не являвшихся студентами СПбГУ ИТМО (данные, о которых отсутствуют в таблице Н\_УЧЕНИКИ). В запросе нельзя использовать DISTINCT.
		\end{enumerate}
		
		\textbf{{\large Запросы на языке SQL:}}
		
		Пункт 1:
		\begin{verbatim}
			SELECT Н_ЛЮДИ.ФАМИЛИЯ, Н_СЕССИЯ.УЧГОД FROM
			Н_СЕССИЯ LEFT JOIN Н_ЛЮДИ ON Н_ЛЮДИ.ИД=Н_СЕССИЯ.ЧЛВК_ИД 
			WHERE Н_ЛЮДИ.ИМЯ='Александр' AND Н_СЕССИЯ.УЧГОД='2001/2002';
		\end{verbatim}
		
		Пункт 2:
		\begin{verbatim}
			SELECT Н_ЛЮДИ.ИД, Н_ОБУЧЕНИЯ.ЧЛВК_ИД, Н_УЧЕНИКИ.ИД FROM Н_ЛЮДИ 
			RIGHT JOIN Н_ОБУЧЕНИЯ ON Н_ЛЮДИ.ИД = Н_ОБУЧЕНИЯ.ЧЛВК_ИД 
			RIGHT JOIN Н_УЧЕНИКИ ON Н_ОБУЧЕНИЯ.ЧЛВК_ИД = Н_УЧЕНИКИ.ЧЛВК_ИД 
			WHERE Н_ЛЮДИ.ОТЧЕСТВО = 'Александрович' AND Н_ОБУЧЕНИЯ.ЧЛВК_ИД > 112514 
			AND Н_УЧЕНИКИ.ИД < 100410;
		\end{verbatim}
		
		Пункт 3:
		\begin{verbatim}
			SELECT CASE
			WHEN COUNT(DATE_PART('year', AGE(CURRENT_DATE, Н_ЛЮДИ.ДАТА_РОЖДЕНИЯ)) > 25)>0 
			THEN 'ДА.' ELSE 'НЕТ.' END AS "Студенты старше 25" 
			FROM Н_ЛЮДИ JOIN Н_УЧЕНИКИ ON Н_УЧЕНИКИ.ЧЛВК_ИД=Н_ЛЮДИ.ИД 
			WHERE Н_УЧЕНИКИ.ГРУППА='3102';
		\end{verbatim}
		
		Пункт 4:
		\begin{verbatim}
			SELECT FOO.ПЛАН_ИД FROM (
			SELECT COUNT(Н_ГРУППЫ_ПЛАНОВ.ПЛАН_ИД) AS КОЛ_ВО, Н_ГРУППЫ_ПЛАНОВ.ПЛАН_ИД 
			FROM Н_ГРУППЫ_ПЛАНОВ JOIN Н_ПЛАНЫ ON Н_ПЛАНЫ.ИД = Н_ГРУППЫ_ПЛАНОВ.ПЛАН_ИД 
			JOIN Н_ОТДЕЛЫ ON Н_ОТДЕЛЫ.ОТД_ИД = Н_ПЛАНЫ.ОТД_ИД 
			WHERE Н_ОТДЕЛЫ.КОРОТКОЕ_ИМЯ='ВТ' 
			GROUP BY Н_ГРУППЫ_ПЛАНОВ.ПЛАН_ИД
			) AS FOO WHERE FOO.КОЛ_ВО>2;
		\end{verbatim}
		
		Пункт 5:
		\begin{verbatim}
			SELECT * FROM (
			SELECT AVG(DATE_PART('year', AGE(CURRENT_DATE, Н_ЛЮДИ.ДАТА_РОЖДЕНИЯ))) AS AGE,
			Н_УЧЕНИКИ.ГРУППА FROM Н_ЛЮДИ 
			JOIN Н_УЧЕНИКИ ON Н_УЧЕНИКИ.ЧЛВК_ИД=Н_ЛЮДИ.ИД 
			GROUP BY Н_УЧЕНИКИ.ГРУППА) AS FOO 
			WHERE FOO.AGE <= (
			SELECT MIN(DATE_PART('year', AGE(CURRENT_DATE, Н_ЛЮДИ.ДАТА_РОЖДЕНИЯ))) AS AGGE 
			FROM Н_ЛЮДИ JOIN Н_УЧЕНИКИ ON Н_УЧЕНИКИ.ЧЛВК_ИД=Н_ЛЮДИ.ИД 
			WHERE Н_УЧЕНИКИ.ГРУППА='1101');
		\end{verbatim}
		
		Пункт 6:
		\begin{verbatim}
			SELECT Н_УЧЕНИКИ.ГРУППА,
			Н_ЛЮДИ.ИД,
			Н_ЛЮДИ.ФАМИЛИЯ,
			Н_ЛЮДИ.ИМЯ,
			Н_ЛЮДИ.ОТЧЕСТВО,
			Н_ФОРМЫ_ОБУЧЕНИЯ.ИД,
			Н_УЧЕНИКИ.СОСТОЯНИЕ 
			FROM Н_УЧЕНИКИ JOIN Н_ПЛАНЫ ON Н_ПЛАНЫ.НАПС_ИД=Н_УЧЕНИКИ.ИД 
			JOIN Н_ФОРМЫ_ОБУЧЕНИЯ ON Н_ФОРМЫ_ОБУЧЕНИЯ.ИД = Н_ПЛАНЫ.ФО_ИД 
			JOIN Н_ОБУЧЕНИЯ ON Н_ОБУЧЕНИЯ.ЧЛВК_ИД=Н_УЧЕНИКИ.ЧЛВК_ИД 
			JOIN Н_ЛЮДИ ON Н_ЛЮДИ.ИД = Н_УЧЕНИКИ.ЧЛВК_ИД 
			JOIN Н_НАПРАВЛЕНИЯ_СПЕЦИАЛ ON Н_НАПРАВЛЕНИЯ_СПЕЦИАЛ.ИД=Н_ПЛАНЫ.НАПС_ИД 
			JOIN Н_НАПР_СПЕЦ ON Н_НАПР_СПЕЦ.ИД=Н_НАПРАВЛЕНИЯ_СПЕЦИАЛ.НС_ИД 
			WHERE Н_УЧЕНИКИ.НАЧАЛО < '2012-09-01' AND 
			(Н_ФОРМЫ_ОБУЧЕНИЯ.НАИМЕНОВАНИЕ='Очная' OR Н_ФОРМЫ_ОБУЧЕНИЯ.НАИМЕНОВАНИЕ='Заочная') 
			AND Н_НАПР_СПЕЦ.КОД_НАПРСПЕЦ='230101';
		\end{verbatim}
		
		Пункт 7:
		\begin{verbatim}
			SELECT Н_ЛЮДИ.ИД, Н_ЛЮДИ.ФАМИЛИЯ, Н_ЛЮДИ.ИМЯ, Н_ЛЮДИ.ОТЧЕСТВО 
			FROM Н_ЛЮДИ 
			WHERE Н_ЛЮДИ.ИД NOT IN (SELECT Н_УЧЕНИКИ.ЧЛВК_ИД FROM Н_УЧЕНИКИ);
		\end{verbatim}
		
		\textbf{{\large Вывод:}}
		В ходе данной лаборатоной я узнал много нового, например такую штуку, как latex... Написал отчет именно в ней и честно говоря я в жизни так не жалел о выборе. Сначала хотел вставить нормальные листинги для запросов, но у listings пакета были другие планы, а именно нет поддержки utf-8, ладно, поставил с помощью танцев с бубном и utf-8. Генерируется pdf, а там \_ и . располагаются крайне странно. Пошел плакать, долго не плакал, так как хотел уже скорее выйти из latex и попробовал minted, но тут надо делать все руками, а не через texmaker, ну и ладно, мне не привыкать и в итоге вообще ничего не отображалось, буквально пара слов на 3 страницах. Пришлось снова плакать, потом подумал использовать verbatim, и тут уже все круто, но я захотел хоть чего-то хорошего от этого проклятого latex и попробовал включать sql скрипты в мой по ссылке в файловой системе. Не работает. Ну и ладно, руками так руками. PS Даже сейчас при компиляции мне latex кидает ошибку за использованный ранее \_, я все \_ что тольео есть рашьне переписывал как надо latex.
	\end{flushleft}
\end{document}